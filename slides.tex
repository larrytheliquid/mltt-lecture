\documentclass[mathserif,usenames,dvipsnames]{beamer}
\usetheme{Warsaw}
\usepackage{mathpartir}
\usepackage{cancel}
\usepackage{url}

\title{Martin-L{\"o}f's Type Theory (MLTT)}
\author{Larry Diehl}
\institute{University of Iowa - Guest Lecture}
\date[April 26, 2018]
{CS:5860 Lambda Calculus and Applications}

\newcommand{\good}[1]{\textcolor{ForestGreen}{#1}}
\newcommand{\bad}[1]{\textcolor{red}{\cancel{#1}}}

\newcommand{\txt}[1]{\textrm{#1}}
\newcommand{\defeq}[0]{\ensuremath{\triangleq}}
\newcommand{\Defeq}[2]{\ensuremath{#1 ~\defeq~ #2}}

\newcommand{\istype}[1]{\ensuremath{#1 ~\textbf{\textrm{type}}}}
\newcommand{\isterm}[2]{\ensuremath{#1 ~\textbf{{:}}~ #2}}

\newcommand{\eqtype}[2]{\ensuremath{#1 ~\textbf{=}~ #2}}
\newcommand{\eqterm}[3]{\ensuremath{#1 ~\textbf{=}~ #2 ~\textbf{{:}}~ #3}}

\newcommand{\hyps}[1]{\ensuremath{~[#1]}}
\newcommand{\hyp}[1]{\hyps{\isterm{x}{#1}}}

\newcommand{\Arr}[2]{\ensuremath{#1 \rightarrow #2}}

\newcommand{\Allv}[2]{\ensuremath{\forall #1. #2}}
\newcommand{\All}[1]{\Allv{X}{#1}}
\newcommand{\allv}[2]{\ensuremath{\Lambda #1. #2}}
\newcommand{\all}[1]{\allv{X}{#1}}

\newcommand{\Funv}[3]{\ensuremath{\Pi #1{:}#2. #3}}
\newcommand{\Fun}[2]{\Funv{x}{#1}{#2}}
\newcommand{\funv}[2]{\ensuremath{\lambda #1. #2}}
\newcommand{\fun}[1]{\funv{x}{#1}}
\newcommand{\app}[2]{\ensuremath{#1~#2}}
\newcommand{\subs}[2]{\ensuremath{#1[#2]}}
\newcommand{\sub}[2]{\ensuremath{#1[#2/x]}}

\newcommand{\Pairv}[3]{\ensuremath{\Sigma #1{:}#2. #3}}
\newcommand{\Pair}[2]{\Pairv{x}{#1}{#2}}
\newcommand{\pair}[2]{(#1, #2)}
\newcommand{\fst}[1]{\ensuremath{\pi_1~#1}}
\newcommand{\snd}[1]{\ensuremath{\pi_2~#1}}

\newcommand{\Type}[0]{\ensuremath{\mathcal{U}}}

\newcommand{\Unit}[0]{\ensuremath{\top}}
\newcommand{\unit}[0]{\ensuremath{\txt{tt}}}
\newcommand{\elimUnit}[2]{\ensuremath{\texttt{elim}_{\Unit}~#1~#2}}
\newcommand{\Bot}[0]{\ensuremath{\bot}}
\newcommand{\elimBot}[1]{\ensuremath{\texttt{elim}_{\Bot}~#1}}

\newcommand{\Truek}[0]{\txt{True}}
\newcommand{\True}[1]{\ensuremath{\Truek~#1}}
\newcommand{\Bool}[0]{\ensuremath{\mathbb{B}}}
\newcommand{\true}[0]{\ensuremath{\txt{tt}}}
\newcommand{\false}[0]{\ensuremath{\txt{ff}}}
\newcommand{\elimBool}[3]{\ensuremath{\texttt{elim}_{\Bool}~#1~#2~#3}}

\newcommand{\Listk}[0]{\ensuremath{\txt{List}}}
\newcommand{\List}[1]{\app{\Listk}{#1}}
\newcommand{\Vectk}[0]{\ensuremath{\txt{Vec}}}
\newcommand{\Vect}[2]{\Veck~#1~#2}

\newcommand{\Nat}[0]{\ensuremath{\mathbb{N}}}
\newcommand{\zero}[0]{\ensuremath{\txt{zero}}}
\newcommand{\suck}[0]{\ensuremath{\txt{suc}}}
\newcommand{\suc}[1]{\ensuremath{\app{\suck}{#1}}}
\newcommand{\elimNatk}[0]{\ensuremath{\txt{elim}_\Nat}}
\newcommand{\elimNat}[3]{\ensuremath{\elimNatk~#1~#2~#3}}
\newcommand{\foldNatk}[0]{\ensuremath{\txt{fold}_\Nat}}

\newcommand{\Wellk}[0]{\ensuremath{\mathcal{W}}}
\newcommand{\Wellv}[3]{\ensuremath{\Wellk #1{:}#2. #3}}
\newcommand{\Well}[2]{\Wellv{x}{#1}{#2}}
\newcommand{\suprk}[0]{\ensuremath{\txt{sup}}}
\newcommand{\supr}[2]{\suprk~#1~#2}
\newcommand{\elimWell}[2]{\ensuremath{\texttt{elim}_{\Wellk}~#1~#2}}

\begin{document}
\frame{\titlepage}

\begin{frame}
\frametitle{Curry-Howard Isomorphism}

\textbf{Propositions as types and proofs as terms}.
Use $\lambda$-calculus as a proof language.
Verified programming by encoding specifications
into types (as logical propositions), and
writing programs that only type-check if they
satisfy the specifications.

MLTT allows any proposition (of intuitionistic logic)
to be represented by a type.
Hence, \textbf{MLTT is a single language
suitable for both programming and theorem proving}.

\end{frame}

\begin{frame}
\frametitle{Curry-Howard Isomorphism}
\framesubtitle{System F}

\begin{center}
\begin{tabular}{ |c||c| } 
 \hline
 Falsity & \Defeq{\Bot}{\All{X}} \\
 \hline
 Truth & \Defeq{\Unit}{\All{\Arr{X}{X}}} \\
 \hline
 Implication & \Defeq{P \Rightarrow Q}{\Arr{P}{Q}} \\ 
 \hline
 F.O. Prop. Quantif. & \Defeq{\Allv{P\!:\!\txt{Prop}}{Q}}{\Allv{P}{Q}} \\ 
 \hline
\end{tabular}
\end{center}

\end{frame}

\begin{frame}
\frametitle{Curry-Howard Isomorphism}
\framesubtitle{System F$_\omega$}

\begin{center}
\begin{tabular}{ |c||c| } 
 \hline
 Falsity & \Defeq{\Bot}{\Allv{X\!:\!\star}{X}} \\
 \hline
 Truth & \Defeq{\Unit}{\Allv{X\!:\!\star}{\Arr{X}{X}}} \\
 \hline
 Negation & \Defeq{\lnot P}{\Arr{P}{\Bot}} \\
 \hline
 Implication & \Defeq{P \Rightarrow Q}{\Arr{P}{Q}} \\ 
 \hline
 Conjuction & \Defeq{P \land Q}{\Allv{X\!:\!\star}{\Arr{(\Arr{P}{\Arr{Q}{X}})}{X}}} \\
 \hline
 Disjunction & \small{\Defeq{P \lor Q}{\Allv{X\!:\!\star}{\Arr{(\Arr{P}{X})}{\Arr{(\Arr{Q}{X})}{X}}}}} \\
 \hline
 F.O. Prop. Quantif. & \Defeq{\Allv{P\!:\!\txt{Prop}}{Q}}{\Allv{P\!:\!\star}{Q}} \\ 
 \hline
 H.O. Prop. Quantif. & \Defeq{\Allv{P\!:\!\txt{Prop}^2}{Q}}{\Allv{P\!:\!\kappa}{Q}} \\
 \hline
\end{tabular}
\end{center}

\end{frame}

\begin{frame}
\frametitle{Curry-Howard Isomorphism}
\framesubtitle{MLTT}

\begin{center}
\begin{tabular}{ |c||c| } 
 \hline
 Falsity & \Bot \\
 \hline
 Truth & \Unit \\
 \hline
 Negation & \Defeq{\lnot P}{\Arr{P}{\Bot}} \\
 \hline
 Implication & \Defeq{P \Rightarrow Q}{\Defeq{\Arr{P}{Q}}{\Fun{P}{Q}}} \\ 
 \hline
 Conjuction & \Defeq{P \land Q}{\Defeq{P \times Q}{\Pair{P}{Q}}} \\
 \hline
 Disjunction & \small{\Defeq{P \lor Q}{\Defeq{P \uplus Q}{\Pair{\Bool}{\elimBool{P}{Q}{x}}}}} \\
 \hline
 F.O. Prop. Quantif. & \Defeq{\Allv{P\!:\!\txt{Prop}}{Q}}{\Funv{P}{\Type}{Q}} \\ 
 \hline
 H.O. Prop. Quantif. & \Defeq{\Allv{P\!:\!\txt{Prop}^2}{Q}}{\Funv{P}{A}{Q}} \\
 \hline
 Term Quantif. & \Defeq{\Allv{t\!:\!A}{Q}}{\Funv{t}{A}{Q}} \\
 \hline
\end{tabular}
\end{center}

\end{frame}

\begin{frame}
\frametitle{Verified Programming}

\begin{block}{Dependently Typed Programming Languages}
Modern dependently typed programming languages, or proof assistants,
implement variations of MLTT (e.g. Agda, Coq, Idris, Lean, etc.).
\end{block}

\begin{block}{Example: Bound-Safe Lookup}
\isterm{\txt{lookup}}{\Funv{A}{\Type}{~\Funv{n}{\Nat}{~\Funv{l}{\List{A}}{~\Arr{n < |l|}{A}}}}}
\end{block}

\begin{block}{Example: Provably Correct Sorting}
$$
\isterm{\txt{sort}}{\Funv{A}{\Type}{~\Arr{\List{A}}{\List{A}}}}
$$
$$
\isterm{\txt{sorted}}{\Funv{A}{\Type}{~\Funv{l}{\List{A}}{~\txt{Ord}~A~(\txt{sort}~A~l) \times \txt{Perm}~A~l~(\txt{sort}~A~l)}}}
$$
\end{block}

\end{frame}

\begin{frame}
\frametitle{MLTT Syntax}

Unified syntax for types and terms,
so typehood and termhood are determined \textit{judgementally}.

\begin{block}{Grammar}
$A,B,a,b,f ::= ...~|~\Fun{A}{B}~|~\fun{b}~|~\app{f}{a}~|~...$
\end{block}

\end{frame}


\begin{frame}
\frametitle{MLTT Judgements}

Static and dynamic semantics are mutually defined judgements.

\begin{block}{Type is well-formed}
\istype{A}
\end{block}

\begin{block}{Term has type}
\isterm{a}{A}
\end{block}

\begin{block}{Type equality}
\eqtype{A}{B}
\end{block}

\begin{block}{Term equality}
\eqterm{a}{a'}{A}
\end{block}

\end{frame}

\begin{frame}
\frametitle{Hypothetical Judgements}
\framesubtitle{or, ``higher-order'' judgements}

\begin{block}{Explicit context}
$\Gamma \vdash \istype{A}$
\end{block}

\begin{block}{Implicit (meta-level) context}
\istype{A}
\end{block}

\end{frame}

\begin{frame}
\frametitle{Hypothetical Judgements}
\framesubtitle{or, ``higher-order'' judgements}

\begin{block}{Explicit context}
$\Gamma, x{:}A \vdash \istype{B}$
\end{block}

\begin{block}{Implicit (meta-level) context}
\istype{B} \hyp{A}
\end{block}

\end{frame}

\begin{frame}
\frametitle{Metatheoretic Properties}

\begin{block}{Type Preservation}
  If \istype{A} and \eqtype{A}{A'}, then \istype{A'}.
  If \isterm{a}{A} and \eqterm{a}{a'}{A}, then \isterm{a'}{A}.
\end{block}

\begin{block}{Consistency}
There is no closed term $e$ s.t. \isterm{e}{\Bot}.
\end{block}

\begin{block}{Normalization}
  If \isterm{a}{A}, then there exists a normal
  $a'$ s.t. \eqterm{a}{a'}{A}.

  If \istype{A}, then there exists a normal
  $A'$ s.t. \eqtype{A}{A'}.
\end{block}

\begin{block}{Decidable Type Checking}
  For all $A$, \istype{A} or its negation.
  For all $a$ and $A$, \isterm{a}{A} or its negation.
  \footnote{
    Assuming intentional equality
    and suitable type annotations on terms.
  }
\end{block}

\end{frame}

\begin{frame}
\frametitle{Equality is an Equivalence Relation}

$$
\inferrule{
  {\istype{A}}
}
{\eqtype{A}{A}}
\qquad
\inferrule{
  {\isterm{a}{A}}
}
{\eqterm{a}{a}{A}}
$$

$$
\inferrule{
  {\eqtype{A}{A'}}
}
{\eqtype{A'}{A}}
\qquad
\inferrule{
  {\eqterm{a}{a'}{A}}
}
{\eqterm{a'}{a}{A}}
$$

$$
\inferrule{
  {\eqtype{A_1}{A_2}}
  \\
  {\eqtype{A_2}{A_3}}
}
{\eqtype{A_1}{A_3}}
\qquad
\inferrule{
  {\eqterm{a_1}{a_2}{A}}
  \\
  {\eqterm{a_2}{a_3}{A}}
}
{\eqterm{a_1}{a_3}{A}}
$$

\end{frame}

\begin{frame}
\frametitle{Package of Rules per Type}
\framesubtitle{e.g. the type of dependent functions}

\begin{block}{Type formation}
e.g. \Fun{A}{B}, and congruences.
\end{block}

\begin{block}{Universe introduction ({\`a} la Russell)}
e.g. \Fun{A}{B}, and congruences.
\end{block}

\begin{block}{Term introduction}
e.g. \fun{b}, and congruences.
\end{block}

\begin{block}{Term elimination}
e.g. \app{f}{a}, and congruences.
\end{block}

\begin{block}{Equality}
  e.g. $\app{(\fun{b})}{a} =_\beta \sub{b}{a}$, and
  $f =_\eta \fun{\app{f}{x}}$.
\end{block}

\end{frame}

\begin{frame}
\frametitle{A Universe (\Type) of Types}

\begin{itemize}

\item
The universe type \Type ~is a safe version of a ``type of types'',
where each type is represented as a \textit{code}
(which is a term) of type \Type.

\item
Each code for a type can be \textit{lifted} to a proper type,
meaning it satisfies the type formation judgement (\istype{A}).

\item
\Type ~is a \textit{predicative} reflection of well-formed types
as terms (typeable by \Type).

\item
Quantifying over universe \textit{codes} (\Type) corresponds to
\textit{predicatively} quantifying over \textit{types}. 

\item
Hence, each \textit{type formation rule} is mirrored
by a \textit{universe introduction rule}.\footnote{
  Except for the \Type-formation rule.
  }

\end{itemize}

\end{frame}

\begin{frame}
\frametitle{Universe ($\Type$ type)}
\framesubtitle{Formation Rules}

$$
\inferrule{
  {}
}
{\istype{\Type}}
\qquad
\inferrule{
  {\isterm{A}{\Type}}
}
{\istype{A}}
$$

$$
\bad{\inferrule{
  {}
}
{\isterm{\Type}{\Type}}}
$$

\end{frame}

\begin{frame}
\frametitle{Predicative Universe}
\framesubtitle{e.g. the type of the identity function}

$$
\istype{\Funv{A}{\Type}{\Arr{A}{A}}}
$$

$$
\bad{\isterm{\Funv{A}{\Type}{\Arr{A}{A}}}{\Type}}
$$

\end{frame}

\begin{frame}
\frametitle{Dependent Functions ($\Pi$ types)}

$$
\inferrule{
  {\istype{A}}
  \\
  {\istype{B} \hyp{A}}
}
{\istype{\Fun{A}{B}}}
\qquad
\inferrule{
  {\isterm{A}{\Type}}
  \\
  {\isterm{B}{\Type} \hyp{A}}
}
{\isterm{\Fun{A}{B}}{\Type}}
$$

$$
\inferrule{
  {\isterm{b}{B} \hyp{A}}
}
{\isterm{\fun{b}}{\Fun{A}{B}}}
\qquad
\inferrule{
  {\isterm{\fun{b}}{\Fun{A}{B}}}
  \\
  {\isterm{a}{A}}
}
{\isterm{\app{f}{a}}{\sub{B}{a}}}
$$

$$
\inferrule{
  {\isterm{a}{A}}
  \\
  {\isterm{b}{B} \hyp{A}}
}
{\eqterm{\app{(\fun{b})}{a}}{\sub{b}{a}}{\sub{B}{a}}}
\qquad
\inferrule{
  {\isterm{f}{\Fun{A}{B}}}
}
{\eqterm{f}{\fun{\app{f}{x}}}{\Fun{A}{B}}}
$$

\end{frame}

\begin{frame}
\frametitle{Dependent Functions ($\Pi$ types)}
\framesubtitle{Congruence Rules (suppressed henceforth)}

$$
\inferrule{
  {\eqtype{A}{A'}}
  \\
  {\eqtype{B}{B'} \hyp{A}}
}
{\eqtype{\Fun{A}{B}}{\Fun{A'}{B'}}}
\qquad
\inferrule{
  {\eqterm{A}{A'}{\Type}}
  \\
  {\eqterm{B}{B'}{\Type} \hyp{A}}
}
{\eqterm{\Fun{A}{B}}{\Fun{A'}{B'}}{\Type}}
$$

$$
\inferrule{
  {\eqterm{b}{b'}{B} \hyp{A}}
}
{\eqterm{\fun{b}}{\fun{b'}}{\Fun{A}{B}}}
\qquad
\inferrule{
  {\eqterm{f}{f'}{\Fun{A}{B}}}
  \\
  {\eqterm{a}{a'}{A}}
}
{\eqterm{\app{f}{a}}{\app{f'}{a'}}{\sub{B}{a}}}
$$

\end{frame}

\begin{frame}
\frametitle{Computation in Types}

\begin{align*}
\txt{Id} &\defeq \funv{A}{A} \\
\txt{id} &\defeq \funv{A}{\funv{a}{a}}
\end{align*}

$$
\isterm{\txt{Id}}{\Arr{\Type}{\Type}}
\qquad
\isterm{\txt{id}}{\Funv{A}{\Type}{\Arr{A}{\app{\txt{Id}}{A}}}}
$$

\end{frame}

\begin{frame}
\frametitle{Conversion Rules}
\framesubtitle{Equality of Types}

$$
\inferrule{
  {\isterm{a}{A}}
  \\
  {\eqtype{A}{A'}}
}
{\isterm{a}{A'}}
\qquad
\inferrule{
  {\eqterm{a}{a'}{A}}
  \\
  {\eqtype{A}{A'}}
}
{\eqterm{a}{a'}{A'}}
$$

\end{frame}

\begin{frame}
\frametitle{Exercise: Derive the Following Typing Judgements}
\framesubtitle{Computation in Types}

$$
\isterm{\funv{A}{A}}{\Arr{\Type}{\Type}}
\qquad
\isterm{\funv{A}{\funv{a}{a}}}{\Funv{A}{\Type}{\Arr{A}{\app{(\funv{X}{X})}{A}}}}
$$

\end{frame}

\begin{frame}
\frametitle{MLTT Open to Extension}

\begin{block}{Core}
Can consider $\Pi$ and \Type~ the ``core'' types of MLTT.
\end{block}

\begin{block}{Open to Extension}
Per Martin-L{\"o}f prefers his theory to be ``open to extension'',
in the sense that adding new types
(like lists, vectors, an internalized equality type, etc.) should
remain possible.\footnote{
  This is why there is no elimination rule for \Type.
} Even axioms (that keep the theory consistent) may be added,
like excluded middle, but at the cost of ``stuck'' computations.
\end{block}

\begin{block}{Minimal Theory for Inductive Types}
Next, we will extend MLTT with a minimal collection of types
that allows inductive types (e.g. $\mathbb{N}$) to be \textit{derived}:
\\\Bot, \Unit, $\Sigma$, \Bool, and $\Wellk$.
\end{block}

\end{frame}

\begin{frame}
\frametitle{Bottom ($\Bot$ type)}

$$
\bad{\inferrule{{}}
{\istype{\Bot}}}
\qquad
\inferrule{{}}
{\isterm{\Bot}{\Type}}
$$

$$
\inferrule{
  {\istype{A}}
  \\
  {\isterm{e}{\Bot}}
}
{\isterm{\elimBot{e}}{A}}
$$

\end{frame}

\begin{frame}
\frametitle{Unit ($\Unit$ type)}

$$
\inferrule{{}}
{\isterm{\Unit}{\Type}}
\qquad
\inferrule{{}}
{\isterm{\unit}{\Unit}}
\qquad
\inferrule{
  {\isterm{u}{\Unit}}
}
{\eqterm{u}{\unit}{\Unit}}
$$

$$
\bad{\inferrule{
  {\istype{P} \hyp{Unit}}
  \\
  {\isterm{p}{\sub{P}{\unit}}}
  \\
  {\isterm{u}{\Unit}}
}
{\isterm{\elimUnit{p}{u}}{\sub{P}{u}}}}
$$

$$
\bad{\inferrule{
  {\istype{P} \hyp{Unit}}
  \\
  {\isterm{p}{\sub{P}{\unit}}}
}
{\eqterm{\elimUnit{p}{\unit}}{p}{\sub{P}{\unit}}}}
$$

\end{frame}

\begin{frame}
\frametitle{Exercise: Solve the Following Terms}
\framesubtitle{Elimination rule for \Unit~is derivable}

\begin{align*}
\txt{ElimEq} &\defeq~? \\
\good{\txt{elimEq}}~ &\good{\defeq~?} \\
\end{align*}

$$
\isterm{\txt{ElimEq}}{?}
\qquad
\good{\isterm{\txt{elimEq}}{?}}
$$

\end{frame}

\begin{frame}
\frametitle{Dependent Pairs ($\Sigma$ types)}

$$
\inferrule{
  {\istype{A}}
  \\
  {\istype{B} \hyp{A}}
}
{\istype{\Pair{A}{B}}}
\qquad
\inferrule{
  {\isterm{A}{\Type}}
  \\
  {\isterm{B}{\Type} \hyp{A}}
}
{\isterm{\Pair{A}{B}}{\Type}}
$$

$$
\inferrule{
  {\istype{B} \hyp{A}}
  \\
  {\isterm{a}{A}}
  \\
  {\isterm{b}{\sub{B}{a}}}
}
{\isterm{\pair{a}{b}}{\Pair{A}{B}}}
$$

$$
\inferrule{
  {\isterm{s}{\Pair{A}{B}}}
}
{\isterm{\fst{s}}{A}}
\qquad
\inferrule{
  {\isterm{s}{\Pair{A}{B}}}
}
{\isterm{\snd{s}}{\sub{B}{\fst{s}}}}
$$

$$
\inferrule{
  {\isterm{s}{\Pair{A}{B}}}
}
{\eqterm{s}{\pair{\fst{s}}{\snd{s}}}{\Pair{A}{B}}}
$$

\end{frame}

\begin{frame}
\frametitle{Booleans ($\Bool$ type)}

$$
\inferrule{{}}
{\isterm{\Bool}{\Type}}
\qquad
\inferrule{{}}
{\isterm{\true}{\Bool}}
\qquad
\inferrule{{}}
{\isterm{\false}{\Bool}}
$$

$$
\inferrule{
  {\istype{P} \hyp{\Bool}}
  \\
  {\isterm{p_t}{\sub{P}{\true}}}
  \\  
  {\isterm{p_f}{\sub{P}{\false}}}
  \\
  {\isterm{b}{\Bool}}
}
{\isterm{\elimBool{p_t}{p_f}{b}}{\sub{P}{b}}}
$$

$$
\inferrule{
  {\istype{P} \hyp{\Bool}}
  \\
  {\isterm{p_t}{\sub{P}{\true}}}
  \\  
  {\isterm{p_f}{\sub{P}{\false}}}
}
{\eqterm{\elimBool{p_t}{p_f}{\true}}{p_t}{\sub{P}{\true}}}
$$

$$
\inferrule{
  {\istype{P} \hyp{\Bool}}
  \\
  {\isterm{p_t}{\sub{P}{\true}}}
  \\  
  {\isterm{p_f}{\sub{P}{\false}}}
}
{\eqterm{\elimBool{p_t}{p_f}{\false}}{p_f}{\sub{P}{\false}}}
$$

\end{frame}

\begin{frame}
\frametitle{Large Eliminations}

\begin{block}{Large Elimination}
A function that computes a type from a term.
\end{block}

\begin{block}{Predicates}
Can be represented as a function (i.e. a large elimination)
from booleans (\Bool) to types. If the predicate is satisfied, the
function returns \Unit (which is inhabited),
otherwise it returns \Bot (which is uninhabited). Predicates can similarly
be defined over other types
(e.g. an \txt{IsSorted} predicate from lists to types).
\end{block}

\begin{block}{Families of Types}
Also called \textit{indexed types}. These are the same as predicates,
but contain additional data of computational value, beyond mere
inhabitance (e.g. a vector of elements indexed by the natural numbers). 
\end{block}

\end{frame}

\begin{frame}
\frametitle{Exercise: Solve the Following Terms}
\framesubtitle{Elimination rule for \Unit~is derivable}

\begin{align*}
\good{\txt{ElimEq}}~ &\good{\defeq~?} \\
\txt{elimEq}~ &\defeq~? \\
\end{align*}

$$
\good{\isterm{\txt{ElimEq}}{?}}
\qquad
\isterm{\txt{elimEq}}{?}
$$

\end{frame}

\begin{frame}
\frametitle{Truth Predicate (\Truek)}

$$
\inferrule{
  {\isterm{b}{\Bool}}
}
{\isterm{\True{b}}{\Type}}
\qquad
\inferrule{{}}
{\eqtype{\True{\true}}{\Unit}}
\qquad
\inferrule{{}}
{\eqtype{\True{\false}}{\Bot}}
$$
\dotfill
$$
\Truek \defeq \funv{b}{\elimBool{\Unit}{\Bot}{b}}
\qquad
\isterm{\Truek}{\Arr{\Bool}{\Type}}
$$

\end{frame}

\begin{frame}
\frametitle{Church-Encoded Datatypes}
\framesubtitle{System F}

\begin{align*}
\Nat &\defeq \All{\Arr{X}{\Arr{(\Arr{X}{X})}{X}}} \\
\zero &\defeq \all{\funv{c_z}{\funv{c_s}{c_z}}} \\
\suck &\defeq \funv{n}{\all{\funv{c_z}{\funv{c_s}{\app{c_s}{n}}}}}
%% \foldNatk &\defeq hm
\end{align*}

$$
\isterm{\Nat}{\star}
\qquad
\isterm{\zero}{\Nat}
\qquad
\isterm{\suck}{\Arr{\Nat}{\Nat}}
$$

%% $$
%% \isterm{\foldNatk}{\All{\Arr{X}{\Arr{(\Arr{X}{X})}{\Arr{\Nat}{X}}}}}
%% $$

\end{frame}

\begin{frame}
\frametitle{Natural Numbers ($\Nat$ type)}

$$
\inferrule{{}}
{\isterm{\Nat}{\Type}}
\qquad
\inferrule{{}}
{\isterm{\zero}{\Nat}}
\qquad
\inferrule{
  {\isterm{n}{\Nat}}
}
{\isterm{\suc{n}}{\Nat}}
$$

$$
\inferrule{
  {\istype{P} \hyp{\Nat}}
  \\\\
  {\isterm{p_z}{\sub{P}{\zero}}}
  \\  
  {\isterm{p_s}{\sub{P}{\suc{n}}} \good{\hyps{\isterm{n}{\Nat}, \isterm{p}{\sub{P}{n}}}}}
  \\
  {\isterm{n}{\Nat}}
}
{\isterm{\elimNat{p_z}{p_s}{n}}{\sub{P}{n}}}
$$

$$
\inferrule{
  {\istype{P} \hyp{\Nat}}
  \\\\
  {\isterm{p_z}{\sub{P}{\zero}}}
  \\  
  {\isterm{p_s}{\sub{P}{\suc{n}}} \hyps{\isterm{n}{\Nat}, \isterm{p}{\sub{P}{n}}}}
}
{\eqterm{\elimNat{p_z}{p_s}{\zero}}{p_z}{\sub{P}{\zero}}}
$$

$$
\inferrule{
  {\istype{P} \hyp{\Nat}}
  \\\\
  {\isterm{p_z}{\sub{P}{\zero}}}
  \\  
  {\isterm{p_s}{\sub{P}{\suc{n}}} \hyps{\isterm{x'}{\Nat}, \isterm{p}{\sub{P}{x'}}}}
  \\
  {\isterm{n}{\Nat}}
}
{\eqterm{\elimNat{p_z}{p_s}{(\suc{n})}}{\good{\subs{p_s}{n/x', (\elimNat{p_z}{p_s}{n})/p}}}{\sub{P}{\suc{n}}}}
$$

\end{frame}

\begin{frame}
\frametitle{W-Encoded Datatypes}
\framesubtitle{MLTT}

\begin{block}{\Wellk~Types (\Well{A}{B})}
\Wellk~ is the type of well-orderings,
or well-founded trees.
The $A$ parameter represents the constructors and their
non-inductive arguments. The dependent $B$ parameter represents
the inductive arguments for each constructor
specified by $A$.
\end{block}

\begin{block}{Inductive Types}
Inductive types (e.g. \Nat) can be encoded by \Wellk, and
their constructors and eliminator (or, induction principle)
can be derived from that of \Wellk~ and the types used for
$A$ and $B$.
\end{block}

\begin{block}{Infinitary Types}
\Wellk~ is the canonical example of an \textit{infinitary type},
or a tree with possibly infinite branches. Instantiating $A$
with \Nat~ results in an infinitely branching tree.
\end{block}

\end{frame}

\begin{frame}
\frametitle{Well-Orderings ($\Wellk$ types)}

$$
\inferrule{
  {\istype{A}}
  \\
  {\istype{B} \hyp{A}}
}
{\istype{\Well{A}{B}}}
\qquad
\inferrule{
  {\isterm{A}{\Type}}
  \\
  {\isterm{B}{\Type} \hyp{A}}
}
{\isterm{\Well{A}{B}}{\Type}}
$$

$$
\inferrule{
  {\istype{B} \hyp{A}}
  \\
  {\isterm{a}{A}}
  \\
  {\isterm{f_b}{\good{\Arr{\sub{B}{a}}{\Well{A}{B}}}}}
}
{\isterm{\supr{a}{f_b}}{\Well{A}{B}}}
$$

$$
\inferrule{
  {\istype{P} \hyp{\Well{A}{B}}}
  \\
  {\tiny{\isterm{p}{\good{\Funv{a}{A}{\Funv{f_b}{(\Arr{\sub{B}{a}}{\Well{A}{B}})}{\Arr{(\Funv{b}{\sub{B}{a}}{\sub{P}{b}})}{\sub{P}{\supr{a}{f_b}}}}}}}}}
  \\
  {\isterm{w}{\Well{A}{B}}}
}
{\isterm{\elimWell{p}{w}}{\sub{P}{w}}}
$$

$$
\inferrule{
  {\istype{P} \hyp{\Well{A}{B}}}
  \\
  {\tiny{\isterm{p}{\Funv{a}{A}{\Funv{f_b}{(\Arr{\sub{B}{a}}{\Well{A}{B}})}{\Arr{(\Funv{b}{\sub{B}{a}}{\sub{P}{b}})}{\sub{P}{\supr{a}{f_b}}}}}}}}
  \\
  {\isterm{a}{A}}
  \\
  {\isterm{f_b}{\Arr{\sub{B}{a}}{\Well{A}{B}}}}
}
{\eqterm{\elimWell{p}{(\supr{a}{f_b}})}{\good{p~a~f_b~(\funv{b}{\elimWell{p}{(f_b~b)}})}}{\sub{P}{w}}}
$$

\end{frame}

\begin{frame}
\frametitle{Natural Numbers ($\Nat$ type)}

\begin{align*}
\Nat &\defeq \Wellv{b}{\Bool}{\True{b}} \\
\zero &\defeq \supr{\false}{(\funv{e}{\elimBot{e}})} \\
\suck &\defeq \funv{n}{\supr{\true}{(\funv{u}{n})}}
%% \elimNat &\defeq \funv{P}{\funv{p_z}{\funv{p_s}{\funv{n}{
%%   \elimBool{x}{y}{(\fst{n})}
%%   }}}}
\end{align*}

$$
\isterm{\Nat}{\star}
\qquad
\isterm{\zero}{\Nat}
\qquad
\isterm{\suck}{\Arr{\Nat}{\Nat}}
$$

\end{frame}

\begin{frame}
\frametitle{Natural Numbers ($\Nat$ type)}

\begin{block}{Caveat}
Deriving typing of \elimNatk~function requires \textit{function extensionality}.

This is needed to identify all possible representations of \zero, and more
specifically all possible terms of type \Arr{\Bot}{\Nat}.
\end{block}

\begin{block}{Extensional Equality of Functions}
$$
\inferrule{
  {\eqterm{f~x}{f'~x}{B} \hyp{A}}
}
{\eqterm{f}{f'}{\Fun{A}{B}}}
$$
\end{block}

\end{frame}

\begin{frame}
\frametitle{Exercise: Solve the Following Terms}
\framesubtitle{Lists and vectors}

\begin{block}{Lists}
What term can be used to derive the type of lists?
$$
\Listk \defeq ~?
\qquad
\isterm{\Listk}{\Arr{\Type}{\Type}}
$$
\end{block}

\begin{block}{Vectors}
What term can be used to derive the type of
vectors (i.e. lists of length $n$)?
$$
\Vectk \defeq ~?
\qquad
\isterm{\Vectk}{\Arr{\Type}{\Arr{\Nat}{\Type}}}
$$
\end{block}

\end{frame}

\begin{frame}
\frametitle{References}

\begin{block}{Book}
\textit{Intuitionistic type theory (1984)}, by
Per Martin-L{\"o}f and Giovanni Sambin.
\end{block}

\begin{block}{Repository of works by Per Martin-L{\"o}f}
\url{https://github.com/michaelt/martin-lof}\\
Includes a searchable and re-typeset version of the book.
\end{block}

\end{frame}


\end{document}

